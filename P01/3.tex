\begin{sol}
\begin{enumerate}[label=\textbf{(\alph*)}]
\item On displacing by a distance $x$ from the equilibrium position, both springs will have a spring force directed towards the equilibrium position. This means that we have $$F_{\text{net}}= -(2Kx + Kx)\hat{x} = -3Kx\hat{x}.$$ 
Now, we write Newton's second law, and get rid of $\hat{x}$ as all the forces are in the $x$-direction to yield
\[m\ddot{x} = -3Kx\implies \omega = \boxed{\sqrt{\frac{3K}{M}}}.\]
\item At mean position, $v = A\omega $. Therefore 
\[A= \frac{v}{\omega} = \boxed{{v}\sqrt{\frac{M}{3K}}}.\]
\item We know $$x = A \sin (\omega t + \phi).$$ From initial conditions we know that $\omega = \sqrt{\frac{3K}{M}}$, $A={v}\sqrt{\frac{M}{3K}}$, and $\phi = 0$. Therefore 
$$ x = \boxed{v\sqrt {\frac {M}{3K}} \sin \left( \sqrt {\frac{3K}{M}} t \right)}.$$
\item From the lecture, we see that we can write the position of the center of mass as the complex function 
\[z (t) = \Re \left[Ae^{i (\omega t + \phi)}\right].\]
From initial conditions we know that $\omega = \sqrt{\frac{3K}{M}}$, $A=v\sqrt{\frac{M}{3K}}$, and $\phi = 0$ and therefore we have,
\[z(t) = \boxed{\Re \left[v\sqrt{\frac{M}{3K}}e^{i\sqrt{\frac{3K}{M}}t}\right]}.\]
\end{enumerate}
\end{sol}