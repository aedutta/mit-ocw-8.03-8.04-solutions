\begin{sol}
\begin{enumerate}[label=\textbf{(\alph*)}]
\item We have two boundary conditions: 
\begin{itemize}
    \item At $x = 0$, $\pdv{\psi (0, t)}{x} = 0$.
    \item At $x = L$, $\pdv{\psi (L, t)}{x} = 0$.
\end{itemize}
By symmetry, the wavefunction is given by 
\[\psi_m (x, t) = A_m \sin (\omega_m t + \beta_m) \sin (k_m x + \alpha_m).\]
Taking the partial derivative of this with respect to $x$ gives us 
\[\pdv{\psi_m}{x} = A_m k_m \sin (\omega_m t + \beta_m)\cos (k_m x + \alpha_m).\]
We now can apply the boundary conditions. 
\begin{itemize}
\item At $x = 0$:
\[\pdv{\psi_m (0, t)}{x} = A_m k_m \sin (\omega_m t + \beta_m) \cos (\alpha_m) = 0.\]
Therefore, we require that 
\[\cos (\alpha_m) = 0\implies \alpha_m = \pi m - \frac{\pi}{2}.\]

\item At $x = L$:
\[\pdv{\psi_m}{x} = A_m k_m \sin (\omega_m t + \beta_m) \cos (k_m L + \pi m - \pi/2) = 0.\]
We require that 
\[k_m L + \pi m - \frac{\pi}{2} = \pi n - \frac{\pi}{2}.\]
This means that 
\[k_m = \frac{\pi (n - m)}{L}, \quad a = n - m = 1, 2, 3, \dots\]
or 
\[k_m = \frac{\pi a}{L}.\]
\end{itemize} 

\item We can draw each mode. 
\begin{itemize}
    \item For the first mode $a = 1$, there w
\end{itemize}
\end{enumerate}
\end{sol}
