\begin{sol}
\begin{enumerate}[label=\textbf{(\alph*)}]
    \item Note that the Hamiltonian energy operator $\hat{H}$ is proportional to the second position derivative of the wavefunction. Differentiating $\Psi$ twice:
    
    \begin{align*}
        \frac{\partial^2\Psi}{\partial x^2} = -\frac{4\pi^2}{a^2}\sqrt{\frac{2}{3a}}\sin{\frac{2\pi x}{a}} - \frac{18\pi^2}{a^2}\sqrt{\frac{1}{3a}}\sin{\frac{3\pi x}{a}}
    \end{align*}
    
    Note that this is \textbf{not} proportional to the initial wavefunction, meaning that it is \textbf{not} an energy eigenstate. Since the Schrodinger equation is linear, we can time evolve the wavefunction as:
    $$\Psi(x,t)=\frac{1}{\sqrt 3}\sqrt{\frac{2}{a}}\sin\left(\frac{2\pi x}{a}\right)e^{-i E_1t}+\sqrt{\frac{2}{3}}\sqrt{\frac{2}{a}}\sin\left(\frac{3\pi x}{a}\right)e^{-i E_2t}$$
    
    \item Note that although the wavefunction itself is not an energy eigenstate, it is composed of two sine functions, both of which are in fact energy eigenstates. Noting that $\hat{H}\Psi = E\Psi$, the energies of each of these eigenstates would be $$E_{\Psi_1}=\frac{4\hbar^2\pi^2}{2ma^2}$$ and $$E_{\Psi_2}=\frac{9\hbar^2\pi^2}{2ma^2}$$ Since we have the superposition of two energy eigenstates, we get:
    $$\Psi = a_1 \Psi_1 + a_2 \Psi_2$$
    and since the wavefunction is normalized, the probabilities to measure the two above energies are $\frac{1}{3}$ and $\frac{2}{3}$ respectively.
    
    \item $\int_{-\infty}^{\infty}$
\end{enumerate}
\end{sol}